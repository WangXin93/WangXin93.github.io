%%%%%%%%%%%%%%%%%%%%%%%%%%%%%%%%%%%%%%%%%
% Medium Length Professional CV
% LaTeX Template
% Version 2.0 (8/5/13)
%
% This template has been downloaded from:
% http://www.LaTeXTemplates.com
%
% Original author:
% Rishi Shah 
%
% Important note:
% This template requires the resume.cls file to be in the same directory as the
% .tex file. The resume.cls file provides the resume style used for structuring the
% document.
%
%%%%%%%%%%%%%%%%%%%%%%%%%%%%%%%%%%%%%%%%%

%----------------------------------------------------------------------------------------
%	PACKAGES AND OTHER DOCUMENT CONFIGURATIONS
%----------------------------------------------------------------------------------------

\documentclass{resume} % Use the custom resume.cls style

\usepackage[left=0.75in,top=0.6in,right=0.75in,bottom=0.6in]{geometry} % Document margins
\usepackage{hyperref}
\newcommand{\tab}[1]{\hspace{.2667\textwidth}\rlap{#1}}
\newcommand{\itab}[1]{\hspace{0em}\rlap{#1}}
\name{Xin Wang} % Your name
\address{Building 3, Number 2999 Nroth Renmin Road, Songjiang,  Shanghai} % Your address
%\address{123 Pleasant Lane \\ City, State 12345} % Your secondary addess (optional)
\address{(+86)133-9148-8435 \\ \href{mailto:wangx@mail.dhu.edu.cn}{wangx@mail.dhu.edu.cn} \\ \href{https://wangxin93.github.io}{wangxin93.github.io}} % Your phone number and email

\begin{document}

%----------------------------------------------------------------------------------------
%	EDUCATION SECTION
%----------------------------------------------------------------------------------------

\begin{rSection}{Education}

{\bf Donghua University, Shanghai, P.R.China} \hfill {\em Sep. 2015 - Pres.} 
\\ Ph.D in Digital Textile Engineering
\\ Related courses: Numerical analysis, Computer graphics, Computer vision
\\ Self-educated: CS61A, CS61B, CS231n, CS20SI, Machine Learning, Python

{\bf University of Manchester, Manchester, United Kingdom} \hfill {\em Dec. 2019 - Aug. 2020} 
\\ Visiting student in Department of Materials
\\ Related courses: \href{https://www.research.manchester.ac.uk/portal/hugh.gong.html}{Fashion recommendation system research}

{\bf Lanzhou University of Technology, Lanzhou, Gansu, P.R.China} \hfill {\em Sep. 2011 - Jul. 2015} 
\\ Bachelor of Textile Engineering.
\\ Related courses: Advanced mathematics, Linear algebra, Probability theory and statistics
%Minor in Linguistics \smallskip \\
%Member of Eta Kappa Nu \\
%Member of Upsilon Pi Epsilon \\


\end{rSection}
%--------------------------------------------------------------------------------
%    Projects And Seminars
%-----------------------------------------------------------------------------------------------
\begin{rSection}{Research}
{\bf Outfit Compatibility Prediction and Diagnosis with \\ Multi-Layered Comparison Network} \hfill {\em Aug. 2018 - Jun. 2019}
\\ - [\href{https://github.com/WangXin93/fashion_compatibility_mcn}{Code}] [\href{https://outfit-diagnosis.herokuapp.com/}{Demo}] - ACM Multimedia 2019, First Author
\begin{itemize}
    \item Propose to diagnose the compatibility of the outfit, which is implemented by using the gradient values to approximate the importance of input similarities.
    \item Propose to learn outfit compatibility from all pairwise similarities.
    \item Leverage the feature hierarchy of CNN to provide both low-level and high-level features for prediction and diagnosis.
\end{itemize}

{\bf Inpainting-based Virtual Try-On Network for Selective \\ Garment Transfer} \hfill {\em Dec. 2018 - Sep. 2019}
\\ - [\href{https://github.com/maktu6/Inpaint-TON}{Code}] - IEEE Access
\begin{itemize}
    \item Propose an Inpainting-based Virtual Try-On Network (I-VTON) which allows the user to try on arbitrary clothes from the model image in a selective manner.
    \item Introduce a skin loss to maintain the skin color of users.
\end{itemize}

{\bf Fabric Idenfication using Convolutional Neural Network} \hfill {\em Oct. 2017 - Jul. 2018}
\\ - [\href{https://github.com/WangXin93/FabricID}{Code}] - Artificial Intelligence on Fashion and Textile Conference (AIFT) 2018, First Author
\begin{itemize}
    \item Explore to retrieve fabric texture with deep extracted features, which is implemented with a CNN with softmax cross entropy and centor loss.
\end{itemize}

\end{rSection}

%----------------------------------------------------------------------------------------
%	WORK EXPERIENCE SECTION
%----------------------------------------------------------------------------------------

\begin{rSection}{Intership}

\begin{rSubsection}{JD AI Research, Beijing, P.R.China}{Aug. 2018 - Mar. 2019}{Computer Vision and Multimedia Lab, R\&D Intern}{}
\item Built a multi-task network for fashion attribute classification, achieved state-of-the-art performance on DeepFashion dataset.
\item Implemented metric learning and sequence models for fashion outfit compatibility prediction, proposed multi-layered comparison network for superior prediction performance and diagnosis ability.
\end{rSubsection}


\end{rSection}


%	EXAMPLE SECTION
%----------------------------------------------------------------------------------------

\begin{rSection}{Selected Awards} 
\item Awarded The First Prize Scholarship for two years during my undergraduate period.
\item Won the \href{https://fashion-challenge.github.io/style-rank.html}{4th} place in 2018 JD fashion style recognition challenge.
\item Awarded JD AI Star Intern in 2019.
\item Awarded Student Travel Grant in ACM Multimedia 2019.
\end{rSection}

%----------------------------------------------------------------------------------------
%	TECHNICAL STRENGTHS SECTION
%----------------------------------------------------------------------------------------

\begin{rSection}{Technical Strengths}

\begin{tabular}{ @{} >{\bfseries}l @{\hspace{6ex}} l }
Computer Language \ & Advanced: Python; Basic: C/C++, Bash, Matlab, SQL \\
Deep Learning Framework \ & TensorFlow, PyTorch \\
Tools \ & Git, Vim, \LaTeX, Scrapy, Scikit-learn, Sed, Awk
\end{tabular}

\end{rSection}

%----------------------------------------------------------------------------------------
%	Personal Interests SECTION
%----------------------------------------------------------------------------------------

% \begin{rSection}{Personal Interests}

% \begin{tabular}{ @{} >{\bfseries}l @{\hspace{6ex}} l }
% Amateur guitar player.
% \end{tabular}

% \end{rSection}


\end{document}
