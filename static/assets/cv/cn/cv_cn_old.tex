%%%%%%%%%%%%%%%%%%%%%%%%%%%%%%%%%%%%%%%%%
% Stylish Curriculum Vitae
% LaTeX Template
% Version 1.0 (18/7/12)
%
% This template has been downloaded from:
% http://www.LaTeXTemplates.com
%
% Original author:
% Stefano (http://stefano.italians.nl/)
%
% IMPORTANT: THIS TEMPLATE NEEDS TO BE COMPILED WITH XeLaTeX
%
% License:
% CC BY-NC-SA 3.0 (http://creativecommons.org/licenses/by-nc-sa/3.0/)
%
% The main font used in this template, Adobe Garamond Pro, does not 
% come with Windows by default. You will need to download it in
% order to get an output as in the preview PDF. Otherwise, change this 
% font to one that does come with Windows or comment out the font line 
% to use the default LaTeX font.
%
%%%%%%%%%%%%%%%%%%%%%%%%%%%%%%%%%%%%%%%%%

\documentclass[a4paper, oneside, final]{scrartcl} % Paper options using the scrartcl class

\usepackage{scrpage2} % Provides headers and footers configuration
\usepackage{titlesec} % Allows creating custom \section's
\usepackage{marvosym} % Allows the use of symbols
\usepackage{tabularx,colortbl} % Advanced table configurations
\usepackage{fontspec} % Allows font customization
\usepackage[UTF8]{ctex} % Allows chinese support
\usepackage[hidelinks]{hyperref} % Allows highlight url and hyperref
\usepackage{fontawesome} % Allow awesome symbols like faGraduationCap

\defaultfontfeatures{Mapping=tex-text}
\setmainfont{Adobe Garamond Pro} % Main document font

\titleformat{\section}{\large\scshape\raggedright}{}{0em}{}[\titlerule] % Section formatting

\pagestyle{scrheadings} % Print the headers and footers on all pages

\addtolength{\voffset}{-0.5in} % Adjust the vertical offset - less whitespace at the top of the page
\addtolength{\textheight}{3cm} % Adjust the text height - less whitespace at the bottom of the page

\newcommand{\gray}{\rowcolor[gray]{.90}} % Custom highlighting for the work experience and education sections

%----------------------------------------------------------------------------------------
% FOOTER SECTION
%----------------------------------------------------------------------------------------

\renewcommand{\headfont}{\normalfont\rmfamily} % Font settings for footer

\cofoot{
\fontsize{12}{17}\selectfont % Letter spacing and font size

上海市 {\large\textperiodcentered} 松江区 {\large\textperiodcentered} 人民北路2999号3号学院楼\\ % Your mailing address
{\Large\Letter} wangxin19930411@163.com \ {\Large\Telefon} (+86) 133-9148-8435 \ 
{\Large\faGithub} github.com/WangXin93% Your email address and phone number
}

%----------------------------------------------------------------------------------------

\begin{document}

\begin{center} % Center everything in the document

%----------------------------------------------------------------------------------------
% HEADER SECTION
%----------------------------------------------------------------------------------------

{\addfontfeature{LetterSpace=20.0}\fontsize{36}{36}\selectfont\scshape 王\ 鑫} % Your name at the top

%----------------------------------------------------------------------------------------
%	OBJECTIVE
%----------------------------------------------------------------------------------------

\section{\faInfo\ 简介}
\begin{flushleft}
我是一名东华大学在读博士生,主要研究方向为服装图像识别与分类及其在电商推荐与搜索当中的应用。希望未来能够从事人工智能研究方向的工作。
\end{flushleft}

%----------------------------------------------------------------------------------------
%	WORK EXPERIENCE
%----------------------------------------------------------------------------------------

\section{\faUsers\ 项目 \& 研究}

\begin{tabularx}{0.97\linewidth}{>{\raggedleft\scshape}p{2cm}X}
\gray 时间 & \textbf{2018年8月 --- 2019年3月}\\
\gray 地点 & \textbf{JD AI Research} \\
\gray 名称 & \textbf{Outfit Compatibility Prediction and Diagnosis with Multi-Layered Comparison Network (ACM MM19 Under Review)}\\
\gray 工具 & \textbf{Python, PyTorch}\\
& 提出一个学习服饰套装组合搭配性(Fashion outfit compatibility)的框架,主要的特点是:(1)从集合中中所有成对单品的相似性预测套装集合的搭配性;(2)利用CNN的层次结构利用从底层到高层的视觉特征来提升搭配性预测的效果;(3)利用反向传播梯度来估计每个输入对搭配性的影响从而找出问题最大的单品来实现搭配性诊断。
\end{tabularx}

\vspace{12pt}

\begin{tabularx}{0.97\linewidth}{>{\raggedleft\scshape}p{2cm}X}
\gray 时间 & \textbf{2017年10月 --- 2018年7月}\\
\gray 名称 & \textbf{Fabric Idenfication using Convolutional Neural Network(\href{https://www.polyu.edu.hk/itc/aift2018/}{AIFT2018})}\\
\gray 工具 & \textbf{Python, TensorFlow}\\
& 使用Softmax Cross Entropy和Centor Loss架构的卷积神经网络提取织物图案特征用于织物图案的检索,获Best Student Paper Award。项目地址:\url{https://github.com/WangXin93/FabricID}
\end{tabularx}

\vspace{12pt}

\begin{tabularx}{0.97\linewidth}{>{\raggedleft\scshape}p{2cm}X}
\gray 时间 & \textbf{2018年3月 --- 2018年7月}\\
\gray 名称 & \textbf{\href{http://fashionai.alibaba.com/}{FashionAI全球挑战赛服饰属性标签识别}}\\
\gray 工具 & \textbf{Python, PyTorch}\\
& 使用卷积神经网络分类服饰属性标签体系中的8种重要属性。在比赛中取得\href{https://tianchi.aliyun.com/competition/rankingList.htm?\&season=0\&raceId=231649\&pageIndex=5}{初赛第87名},复赛第66名的成绩。项目地址:\url{https://github.com/WangXin93/torchfashion}
\end{tabularx}

\vspace{12pt}

\begin{tabularx}{0.97\linewidth}{>{\raggedleft\scshape}p{2cm}X}
\gray 时间 & \textbf{2018年7月 --- 2018年7月}\\
\gray 名称 & \textbf{\href{https://fashion-challenge.github.io/}{JD AI Fashion-Challenge时尚风格识别}}\\
\gray 工具 & \textbf{Python, PyTorch}\\
& 对女装商品照片的13种女装风格进行多标签预测,使用阈值寻优和\href{https://github.com/timgaripov/swa}{SWA}提高性能表现。在比赛中取得\href{https://fashion-challenge.github.io/style-rank.html}{第4名的成绩}。
\end{tabularx}

%----------------------------------------------------------------------------------------
%	EDUCATION
%----------------------------------------------------------------------------------------

\section{\faGraduationCap\ 教育经历}

\begin{tabularx}{0.97\linewidth}{>{\raggedleft\scshape}p{2cm}X}
\gray 时间 & \textbf{2015年9月 --- 至今}\\
\gray 专业 & \textbf{数字化纺织工程(博士)}\\
\gray 课程 & \textbf{数值分析,计算机图形学,图像处理,自修\href{https://github.com/WangXin93/CS231n-Spring-2017-Assignment}{CS231n},\href{https://github.com/WangXin93/CS20SI-my-example}{CS20SI}, \href{https://github.com/WangXin93/machine_learning}{Machine Learning},\href{https://github.com/WangXin93/My_python_demo}{Python}}\\
\gray 学校 & \textbf{东华大学} \hfill 上海, 中国\\
%& Extra information about degree
\end{tabularx}

\vspace{12pt}

\begin{tabularx}{0.97\linewidth}{>{\raggedleft\scshape}p{2cm}X}
\gray 时间 & \textbf{2011年9月 --- 2015年7月}\\
\gray 专业 & \textbf{纺织工程(本科)}\\
\gray 课程 & \textbf{高等数学,线性代数,概率论与数理统计}\\
\gray 学校 & \textbf{兰州理工大学} \hfill 甘肃, 中国\\
%& Extra information about degree
\end{tabularx}

%----------------------------------------------------------------------------------------
%	SKILLS
%----------------------------------------------------------------------------------------

\section{\faCogs\ 技能}

\begin{tabular}{ @{} >{\bfseries}l @{\hspace{6ex}} l }
计算机语言 & 熟练Python, 了解C/C++, Bash, Matlab, SQL \\
深度学习框架 & TensorFlow, PyTorch \\
工具 & Git, Vim, \LaTeX, Scrapy, Scikit-learn, Sed, Awk
\end{tabular}

%----------------------------------------------------------------------------------------

\end{center}

\end{document}